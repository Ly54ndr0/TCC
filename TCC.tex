%% abtex2-modelo-trabalho-academico.tex, v-1.9.7 laurocesar
%% Copyright 2012-2018 by abnTeX2 group at http://www.abntex.net.br/ 
%%
%% This work may be distributed and/or modified under the
%% conditions of the LaTeX Project Public License, either version 1.3
%% of this license or (at your option) any later version.
%% The latest version of this license is in
%%   http://www.latex-project.org/lppl.txt
%% and version 1.3 or later is part of all distributions of LaTeX
%% version 2005/12/01 or later.
%%
%% This work has the LPPL maintenance status `maintained'.
%% 
%% The Current Maintainer of this work is the abnTeX2 team, led
%% by Lauro César Araujo. Further information are available on 
%% http://www.abntex.net.br/
%%
%% This work consists of the files abntex2-modelo-trabalho-academico.tex,
%% abntex2-modelo-include-comandos and abntex2-modelo-references.bib
%%

% ------------------------------------------------------------------------
% ------------------------------------------------------------------------
% abnTeX2: Modelo de Trabalho Academico (tese de doutorado, dissertacao de
% mestrado e trabalhos monograficos em geral) em conformidade com 
% ABNT NBR 14724:2011: Informacao e documentacao - Trabalhos academicos -
% Apresentacao
% ------------------------------------------------------------------------
% ------------------------------------------------------------------------

\documentclass[
	% -- opções da classe memoir --
	12pt,				% tamanho da fonte
	openright,			% capítulos começam em pág ímpar (insere página vazia caso preciso)
	twoside,			% para impressão em recto e verso. Oposto a oneside
	a4paper,			% tamanho do papel. 
	% -- opções da classe abntex2 --
	%chapter=TITLE,		% títulos de capítulos convertidos em letras maiúsculas
	%section=TITLE,		% títulos de seções convertidos em letras maiúsculas
	%subsection=TITLE,	% títulos de subseções convertidos em letras maiúsculas
	%subsubsection=TITLE,% títulos de subsubseções convertidos em letras maiúsculas
	% -- opções do pacote babel --
	english,			% idioma adicional para hifenização
	french,				% idioma adicional para hifenização
	spanish,			% idioma adicional para hifenização
	brazil				% o último idioma é o principal do documento
	]{abntex2}

% ---
% Pacotes básicos 
% ---
\usepackage{lmodern}			% Usa a fonte Latin Modern			
\usepackage[T1]{fontenc}		% Selecao de codigos de fonte.
\usepackage[utf8]{inputenc}		% Codificacao do documento (conversão automática dos acentos)
\usepackage{indentfirst}		% Indenta o primeiro parágrafo de cada seção.
\usepackage{color}				% Controle das cores
\usepackage{graphicx}			% Inclusão de gráficos
\usepackage{microtype} 			% para melhorias de justificação
% ---
		
% ---
% Pacotes adicionais, usados apenas no âmbito do Modelo Canônico do abnteX2
% ---
\usepackage{lipsum}				% para geração de dummy text
\usepackage{enumerate}
\usepackage{amsmath, amsthm, amssymb, amsfonts}
% ---

% ---
% Pacotes de citações
% ---
\usepackage[brazilian,hyperpageref]{backref}	 % Paginas com as citações na bibl
\usepackage[alf]{abntex2cite}	% Citações padrão ABNT
\usepackage{subfig}
% --- 
% CONFIGURAÇÕES DE PACOTES
% --- 

% ---
% Configurações do pacote backref
% Usado sem a opção hyperpageref de backref
\renewcommand{\backrefpagesname}{Citado na(s) página(s):~}
% Texto padrão antes do número das páginas
\renewcommand{\backref}{}
% Define os textos da citação
\renewcommand*{\backrefalt}[4]{
	\ifcase #1 %
		Nenhuma citação no texto.%
	\or
		Citado na página #2.%
	\else
		Citado #1 vezes nas páginas #2.%
	\fi}%
% ---

% ---
% Informações de dados para CAPA e FOLHA DE ROSTO
% ---
\titulo{Estudo de Oscilações de Estados em Autômatos Celulares com Inércia}
\autor{Ly Sandro Amorim de Campos Salles}
\local{Curitiba}
\data{2019}
\orientador{Marlus Koehler}
\coorientador{}
\instituicao{%
  Universidade Federal do Paraná -- UFPR
  \par
  Departamento de Física}
\tipotrabalho{Trabalho de Conclusão de Curso}
% O preambulo deve conter o tipo do trabalho, o objetivo, 
% o nome da instituição e a área de concentração 
\preambulo{Trabalho de Conclusão de Curso apresentado à banca examinadora como requisito para a aprovação na disciplica TCC-B (CF1811) do curso de Licenciatura em Física da Universidade Federal do Paraná.}
% ---


% ---
% Configurações de aparência do PDF final

% alterando o aspecto da cor azul
\definecolor{blue}{RGB}{0,0,0}

% informações do PDF
\makeatletter
\hypersetup{
     	%pagebackref=true,
		pdftitle={\@title}, 
		pdfauthor={\@author},
    	pdfsubject={\imprimirpreambulo},
	    pdfcreator={LaTeX with abnTeX2},
		pdfkeywords={abnt}{latex}{abntex}{abntex2}{trabalho acadêmico}, 
		colorlinks=true,       		% false: boxed links; true: colored links
    	linkcolor=blue,          	% color of internal links
    	citecolor=blue,        		% color of links to bibliography
    	filecolor=magenta,      		% color of file links
		urlcolor=blue,
		bookmarksdepth=4
}
\makeatother
% --- 

% ---
% Posiciona figuras e tabelas no topo da página quando adicionadas sozinhas
% em um página em branco. Ver https://github.com/abntex/abntex2/issues/170
\makeatletter
\setlength{\@fptop}{5pt} % Set distance from top of page to first float
\makeatother
% ---

% ---
% Possibilita criação de Quadros e Lista de quadros.
% Ver https://github.com/abntex/abntex2/issues/176
%
\newcommand{\quadroname}{Quadro}
\newcommand{\listofquadrosname}{Lista de quadros}

\newfloat[chapter]{quadro}{loq}{\quadroname}
\newlistof{listofquadros}{loq}{\listofquadrosname}
\newlistentry{quadro}{loq}{0}

% configurações para atender às regras da ABNT
\setfloatadjustment{quadro}{\centering}
\counterwithout{quadro}{chapter}
\renewcommand{\cftquadroname}{\quadroname\space} 
\renewcommand*{\cftquadroaftersnum}{\hfill--\hfill}

\setfloatlocations{quadro}{hbtp} % Ver https://github.com/abntex/abntex2/issues/176
% ---

% --- 
% Espaçamentos entre linhas e parágrafos 
% --- 

% O tamanho do parágrafo é dado por:
\setlength{\parindent}{1.3cm}

% Controle do espaçamento entre um parágrafo e outro:
\setlength{\parskip}{0.2cm}  % tente também \onelineskip

% ---
% compila o indice
% ---
\makeindex
% ---

% ----
% Início do documento
% ----
\begin{document}

% Seleciona o idioma do documento (conforme pacotes do babel)
%\selectlanguage{english}
\selectlanguage{brazil}

% Retira espaço extra obsoleto entre as frases.
\frenchspacing 

% ----------------------------------------------------------
% ELEMENTOS PRÉ-TEXTUAIS
% ----------------------------------------------------------
% \pretextual

% ---
% Capa
% ---
\imprimircapa
% ---

% ---
% Folha de rosto
% (o * indica que haverá a ficha bibliográfica)
% ---
\imprimirfolhaderosto*
% ---

% ---
% Inserir a ficha bibliografica
% ---

% Isto é um exemplo de Ficha Catalográfica, ou ``Dados internacionais de
% catalogação-na-publicação''. Você pode utilizar este modelo como referência. 
% Porém, provavelmente a biblioteca da sua universidade lhe fornecerá um PDF
% com a ficha catalográfica definitiva após a defesa do trabalho. Quando estiver
% com o documento, salve-o como PDF no diretório do seu projeto e substitua todo
% o conteúdo de implementação deste arquivo pelo comando abaixo:
%
% \begin{fichacatalografica}
%     \includepdf{fig_ficha_catalografica.pdf}
% \end{fichacatalografica}

\begin{fichacatalografica}
	\sffamily
	\vspace*{\fill}					% Posição vertical
	\begin{center}					% Minipage Centralizado
	\fbox{\begin{minipage}[c][8cm]{13.5cm}		% Largura
	\small
	\imprimirautor
	%Sobrenome, Nome do autor
	
	\hspace{0.5cm} \imprimirtitulo  / \imprimirautor. --
	\imprimirlocal, \imprimirdata-
	
	\hspace{0.5cm} \thelastpage p. : il. (algumas color.) ; 30 cm.\\
	
	\hspace{0.5cm} \imprimirorientadorRotulo~\imprimirorientador\\
	
	\hspace{0.5cm}
	\parbox[t]{\textwidth}{\imprimirtipotrabalho~--~\imprimirinstituicao,
	\imprimirdata.}\\
	
	\hspace{0.5cm}
		1. Autômato Celular.
		2. Aglomeração.
    2. Liquidez.
    3. Volatilidade.
    4. Estocasticidade.
    5. Modelagem.
    6. Econofísica.
    7. Caos.
		I. Marlus Koehler.
		II. Universidade Federal do Paraná.
		III. Departamento de Física.
		IV. Estudo de Oscilações de Estados em Autômatos Celulares com Inércia.
	\end{minipage}}
	\end{center}
\end{fichacatalografica}
% ---

% ---
% Inserir errata
% ---
%\begin{errata}
%Elemento opcional da \citeonline[4.2.1.2]{NBR14724:2011}. Exemplo:
%
%\vspace{\onelineskip}
%
%FERRIGNO, C. R. A. \textbf{Tratamento de neoplasias ósseas apendiculares com
%reimplantação de enxerto ósseo autólogo autoclavado associado ao plasma
%rico em plaquetas}: estudo crítico na cirurgia de preservação de membro em
%cães. 2011. 128 f. Tese (Livre-Docência) - Faculdade de Medicina Veterinária e
%Zootecnia, Universidade de São Paulo, São Paulo, 2011.
%
%\begin{table}[htb]
%\center
%\footnotesize
%\begin{tabular}{|p{1.4cm}|p{1cm}|p{3cm}|p{3cm}|}
%  \hline
%   \textbf{Folha} & \textbf{Linha}  & \textbf{Onde se lê}  & \textbf{Leia-se}  \\
%    \hline
%    1 & 10 & auto-conclavo & autoconclavo\\
%   \hline
%\end{tabular}
%\end{table}
%
%\end{errata}
% ---

% ---
% Inserir folha de aprovação
% ---

% Isto é um exemplo de Folha de aprovação, elemento obrigatório da NBR
% 14724/2011 (seção 4.2.1.3). Você pode utilizar este modelo até a aprovação
% do trabalho. Após isso, substitua todo o conteúdo deste arquivo por uma
% imagem da página assinada pela banca com o comando abaixo:
%
% \begin{folhadeaprovacao}
% \includepdf{folhadeaprovacao_final.pdf}
% \end{folhadeaprovacao}
%
%\begin{folhadeaprovacao}

%  \begin{center}
%    {\ABNTEXchapterfont\large\imprimirautor}

%    \vspace*{\fill}\vspace*{\fill}
%    \begin{center}
%      \ABNTEXchapterfont\bfseries\Large\imprimirtitulo
%    \end{center}
%    \vspace*{\fill}
%    
%    \hspace{.45\textwidth}
%    \begin{minipage}{.5\textwidth}
%        \imprimirpreambulo
%    \end{minipage}%
%    \vspace*{\fill}
%   \end{center}
%        
%   Trabalho aprovado. \imprimirlocal,  de novembro de 2012:

%   %\assinatura{\textbf{\imprimirorientador} \\ Orientador} 
%   \assinatura{\textbf{Professor} \\ Convidado 1}
%   \assinatura{\textbf{Professor} \\ Convidado 2}
%   \assinatura{\textbf{Professor} \\ Convidado 3}
%   %\assinatura{\textbf{Professor} \\ Convidado 4}
%      
%   \begin{center}
%    \vspace*{0.5cm}
%    {\large\imprimirlocal}
%    \par
%    {\large\imprimirdata}
%    \vspace*{1cm}
%  \end{center}
%  
%\end{folhadeaprovacao}
% ---

% ---
% Dedicatória
% ---
%\begin{dedicatoria}
%   \vspace*{\fill}
%   \centering
%   \noindent
%   \textit{ Este trabalho é dedicado às crianças adultas que,\\
%   quando pequenas, sonharam em se tornar cientistas.} \vspace*{\fill}
%\end{dedicatoria}
% ---

% ---
% Agradecimentos
% ---
\begin{agradecimentos}
Agradeço ao meu Orientador Marlus Koehler pela liberdade e pela confiança no meu trabalho.

\end{agradecimentos}
% ---

% ---
% Epígrafe
% ---
%\begin{epigrafe}
%    \vspace*{\fill}
%	\begin{flushright}
%		\textit{``Texto\\
%		Tente outra vez\\
%		''\\
%		(Livro das Virtudes para Crianças)}
%	\end{flushright}
%\end{epigrafe}
% ---

% ---
% RESUMOS
% ---

% resumo em português
\setlength{\absparsep}{18pt} % ajusta o espaçamento dos parágrafos do resumo
\begin{resumo}
  Utilizando um autômato celular bidimensional desenvolvido por Dietrich Stauffer e Gérard Weisbuch em 2002 para simulações de agentes em estado de compra ou venda em um sistema, determinamos a intensidade com que esses agentes tendem a tomar decisões em conjunto, denominada afinidade. Isso foi feito considerando vizinhanças de Moore e Von Neumann. Desenvolvemos duas interpretações para a variável que determina a velocidade do sistema: liquidez, e volatilidade. Essas simulações foram feitas para números diferentes de agentes, variando de 2500 a 250000. Descobrimos, nessa análise positiva, que a afinidade é uma função sigmóide da liquidez, variando com o número de agentes. Com base nesses dados percebemos que, para sistemas com alta liquidez, como o mercado de alimentos na vida real, a tendência de aglomeração de agentes é alta, o que pode explicar a existência das Centrais de Abastecimento CEASA. Analogamente, quando a liquidez é baixa, como no caso das negociações envolvendo itens de colecionador e figurinhas de copa do mundo, a tendência de aglomeração é menor, fazendo com que existam mais aglomerados esparsamente distribuídos, como grupos de troca de figurinhas em várias praças de uma mesma cidade. Com a análise de volatilidade foi percebido um comportamento semelhante ao de mercados financeiros, sendo a oscilação do preço do produto defasada em relação à oscilação do número de compradores. Aproveitando a forma com que computadores geram números aleatórios também foi possível verificar se os autômatos celulares estudados apresentavam comportamento caótico. Colateralmente foi desenvolvido um algoritmo de contagem de aglomerados para autômatos celulares em duas dimensões que se mostrou mais eficiente do que os utilizados atualmente ao considerar um algoritmo semelhante ao processo de contaminação celular.

 \textbf{Palavras-chave}: Autômato celular. Aglomeração. Liquidez. Volatilidade. Estocasticidade. Modelagem. Econofísica. Caos.
\end{resumo}

% resumo em inglês
\begin{resumo}[Abstract]
 \begin{otherlanguage*}{english}
   Using a bidimensional cellular automata developed by Dietrich Stauffer and Gérard Weisbuch in 2002 to run simulations of agents in states of buying and selling in a system, we determined the intensity with which those agents make decision in groups, and we named it affinity. That was done considering Moore's and VonNeumann's neighboorhoods. We developed two interpretations for the variable responsible for determining the velocity of the system: liquidity and volatility. The simulations were done with different numbers of agents, varying from 2500 to 250000. We discovered, in this positive analysis, that the affinity is a sigmoid function of the liquidity, changing with the number of agents in the system. Based on the data obtained we noticed that, for systems with high liquidity, as the foods market, the tendency for agents to form clusters is high, what could explain Supply Centrals like the CEASAs found in Brazil. Similarly, when the liquidity is low, as the collectors items, the tendency for the agents to make clusters is smaller, making smaller dispersed clusters. With the analysis of volatility we noticed a behavior alike to the stock market, having a buying cycle desynchronized to the price cycle. Seizing the way computers generate random numbers, it was also possible to verify if the cellular automata studied were chaotic. As an additional feature, we developed an algorithm for counting clusters in two dimentions matrixes that was demonstrated to be faster than other similar algorithms.

   \vspace{\onelineskip}
 
   \noindent 
   \textbf{Keywords}: Cellular automata. Clustering. Liquidity. Volatility. Stochasticity. Modeling. Econophysics. Chaos.
 \end{otherlanguage*}
\end{resumo}
% ---

% ---
% inserir lista de ilustrações
% ---
\pdfbookmark[0]{\listfigurename}{lof}
\listoffigures*
\cleardoublepage
% ---

% ---
% inserir lista de quadros
% ---
%\pdfbookmark[0]{\listofquadrosname}{loq}
%\listofquadros*
%\cleardoublepage
% ---

% ---
% inserir lista de tabelas
% ---
\pdfbookmark[0]{\listtablename}{lot}
\listoftables*
\cleardoublepage
% ---

% ---
% inserir lista de abreviaturas e siglas
% ---
\begin{siglas}
  \item[UFPR] Universidade Federal do Paraná
  \item[ICA ou INCA] Inhomogenous Cellular Automata
\end{siglas}
% ---

% ---
% inserir lista de símbolos
% ---
\begin{simbolos}
  \item[$ \forall $] Para todo
  \item[$ \Rightarrow $] Implica
  \item[$ \Leftrightarrow $] Se, e somente se
  \item[$ \in $] Pertence
\end{simbolos}
% ---

% ---
% inserir o sumario
% ---
\pdfbookmark[0]{\contentsname}{toc}
\tableofcontents*
\cleardoublepage
% ---



% ----------------------------------------------------------
% ELEMENTOS TEXTUAIS
% ----------------------------------------------------------
\textual

% ----------------------------------------------------------
% Introdução (exemplo de capítulo sem numeração, mas presente no Sumário)
% ----------------------------------------------------------
\chapter{Introdução}
% ----------------------------------------------------------

Autômatos são objetos que operam a si mesmos. Essa definição, disponível na Encyclopaedia Britannica (Referências \cite{britannica1} e \cite{britannica2}), traz a possibilidade de objetos de uso cotidiano, como o computador e o celular, se encaixarem na categoria de autômatos. Porém, a existência desses objetos não é nova, existindo autômatos desde a Grécia antiga na figura de um modelo de madeira de um pombo construído por Archytas de Tarentum \cite{britannica1}. Utilizações contemporâneas de autômatos incluem as redes neurais e as inteligências artificiais. Também existem outros tipos famosos de autômatos, como os autômatos celulares.

Os autômatos celulares são, segundo \citeonline{britannica3}, simples modelos espacialmente distribuídos capazes de simular processos do mundo real. Eles foram inventados por John von Neumann e Stanislaw Ulam no Laboratório Nacional de Los Alamos em 1940 e ficaram famosos através do ``Game of Life'', inventado por John Conway em 1970, que simula a dinâmica de vida, morte e população.

\section*{A nova ciência de Stephen Wolfram}

No livro \textit{A New Kind Of Science}, o autor Stephen Wolfram \cite{wolfram} determinou alguns axiomas sobre os autômatos, que basicamente são os seguintes:
\begin{enumerate}
	\item É possível criar sistemas complexos a partir de regras simples.
	\item Princípio da equivalência computacional: para todo comportamento que não é obviamente simples  existe uma computação correspondente equivalentemente sofisticada.
	\item É possível descobrir como um sistema vai se comportar simplesmente executando o experimento e observando o que acontece. Porém, o sucesso das ciências aconteceu ao serem encontradas fórmulas matemáticas que previam os resultados desses experimentos sem a necessidade da execução do mesmo. Apesar de o princípio da equivalência computacional garantir uma resposta mais complexa, soluções mais simples podem existir.
	\item Apesar da possibilidade de sistemas simplificáveis, podem existir sistemas irredutíveis computacionalmente, ou seja, cuja única forma de prever o resultado do experimento é executando o experimento.
\end{enumerate}
Este último axioma é o que causou Stephen a desenvolver a teoria de autômatos celulares descrita no livro dele.

\section*{O autômato estocástico de Stauffer}

Em 2003, \citeonline{stauffer} publicou um artigo no qual ele descreveu um autômato celular não-homogêneo (chamado por ele de InCA ou Inhomogeneous Cellullar Automata) no qual a atualização das células ocorria de forma aleatória e cada célula tinha uma espécie de resistência interna a mudar de estado. Formalmente, em uma matriz, cada célula do autômato de Stauffer guarda dois números: o próprio estado e o próprio limiar. O limiar é definido como sendo o menor valor da soma dos estados das células vizinhas necessário para que a célula fique no estado $+1$ caso ela seja atualizada. Esse número pode ser positivo ou negativo. Caso a célula seja atualizada mas ela não tenha um número de vizinhos no estado $+1$ suficiente para ficar no estado $+1$, a célula fica no estado $-1$ e seu limiar diminui por um número aleatório entre $0$ e $q$, onde $q$ é o ajuste máximo de limiar, definido antes do início da simulação. Caso a célula fique no estado $+1$ ao ser atualizada, o seu limiar aumenta por um número aleatório entre $0$ e $q$.

Em seu artigo ``Adjustment and social choice'', \citeonline{stauffer} explorou o comportamento das oscilações do estado médio e do limiar médio da matriz em função do tempo de simulação. Nisso ele percebeu que, quanto maior o valor do ajuste máximo de limiar $q$, maior é a frequência dessas oscilações em função do tempo. Além de comentar sobre a possibilidade de aplicação em Física de Transição de Fases, Stauffer utilizou os resultados obtidos para propor previsões e modelos para mercados financeiros.

\section*{A dinâmica de padrões de Klaus Kramer}

Já em 2014, \citeonline{klaus} desenvolveu estudos sobre autômatos celulares envolvendo três estados e focou na formação de aglomerados. O autômato celular dele funciona, de forma sintética, assim: O Autômato Celular tem três possíveis estados para cada célula (\texttt{-1}, \texttt{0} e \texttt{+1}); as células estão organizadas em uma malha quadrada; as oito celulas adjacentes formam a vizinhança de cada célula; a matriz de células é finita; células nas bordas interagem somente com as células nas bordas ou no interior da matriz; cada célula tem uma resistência à mudança (ou inércia) cujo valor fica entre $0$ e $8$, sendo $0$ equivalente a nenhuma resistência e $8$ equivalente a resistência total; existe uma chance de cada célula que tem uma vizinha no estado $0$ ser convertida para o estado $0$; ao ser atualizada, se a a célula tiver uma inércia maior que o valor absoluto da soma dos estados das células na vizinhança dela, a célula continua no mesmo estado; caso o valor da inércia da célula seja menor que o valor absoluto da soma da vizinhança, a célula muda para o estado \texttt{-1} ou \texttt{+1} dependendo se a soma é menor ou maior que $0$, respectivamente; se o valor da soma for zero, a célula continua no mesmo estado. Esse processo era repetido até que uma configuração estácionária fosse alcançada. Uma configuração era considerada estacionária se a média dos estados das últimas dez gerações estivesse dentro de $0.001\%$ da média dos estados dessa configuração.

Durante a análise, o valor da inércia foi igual para todas as células e não mudou durante as simulações. Os seguintes parâmetros foram utilizados: a densidade de cada estado, o tamanho médio dos aglomerados, o número de aglomerados, e o tempo de convergência para uma disposição estacionária. As matrizes iniciais foram criada parcialmente de forma aleatória, sendo que a densidade do estado \texttt{0} foi forçada para um valor pré-definido e o estado \texttt{+1} foi forçado a ser o dominante. O espaço de amostra para as simulações foi escolhido pelo autor: os conjuntos de dados escolhidos tinham uma matriz inicial com tamanho médio de aglomerados entre $2.4$ e $2.8$. Isso foi feito porque ``os valores resultaram numa melhor distribuição dos dados iniciais sobre o espaço de parâmetros''.

Aglomerados foram considerados como sendo formados por duas ou mais células adjacentes ortogonalmente no mesmo estado. 
Para a contagem dos aglomerados foi utilizado o algoritmo de Hoshen-Kopelman, que consiste em encontrar aglomerados, etiquetar os aglomerados encontrados, procurar aglomerados adjacentes, realizar a união desses aglomerados, e iterar os dois últimos passos até que não existam mais aglomerados adjacentes. A contagem de aglomerados foi implementada com o objetivo de estudar qualitativamente a evolução da distribuição espacial dos estados do sistema.

Foi observado, para todos os valores de inércia, que a densidade de células positivas aumentou até o estado estacionário. A relação entre densidade inicial de células positivas e densidade final de células positivas ficou localmente linear no intervalo $[0.33, 0.36]$, sendo que densidades maiores de células no estado positivo resultaram na diminuição de aglomerados positivos.

Também foi estudado, para larguras de matriz iguais a $20$ $40$ $60$ $100$ e $200$ células, qual a influência do tamanho da matriz sobre a dinâmica do sistema. Foi concluído que a densidade final de células no estado \texttt{+1} independe do tamanho do sistema.

Sobre um comportamento anômalo para quando a inércia é igual à metade do número de células na vizinhança de cada célula, foi explicado que isso é devido ao fato de as mudanças de estado serem menos frequêntes quando isso acontece já que as células das bordas nunca atualizam nesse caso. Com isso, como as bordas têm muita influência em matrizes pequenas, o comportamento anômalo é criado.

Finalmente, no estudo de ecótonos (região de sobrevivência de um estado/bioma entre a fronteira competitiva de outros dois estados/biomas), foi estudada a evolução de uma única matriz inicial considerando vários parâmetros diferentes. O objetivo foi mostrar que a emergência de ecótonos pode ser explicada através de um autômato celular. Comportamentos semelhantes aos reais foram observados para valores pequenos de inércia (0, 1 e 2). Quando o valor da inércia foi definido como $3$, houve a situação de o ecótono tomar conta de um dos outros dois biomas, quebrando o equilíbrio. Foi concluído que a única possibilidade de haver a expansão do estado \texttt{0} é se houver uma chance não-nula de qualquer célula nos outros estados com pelo menos uma célula vizinha no estado \texttt{0} mudar para o estado \texttt{0}.

\section*{Objetivos}


\subsection*{Objetivo geral:}
Explorar a dinâmica de estados do autômato celular de \citeonline{stauffer}, juntando a ele a ideia de estudar aglomerados empreendida por \citeonline{klaus}, a fim de buscar padrões que possam ser associados a fenômenos naturais.

\subsection*{Objetivos específicos:}
\begin{enumerate}
  \item Desenvolver um autômato celular capaz de executar simulações semelhantes às desenvolvidas por \citeonline{stauffer};
  \item Implementar um algoritmo de contagem de aglomerados no autôato celular, como fez \citeonline{klaus};
  \item Desenvolver simulações para vizinhanças de VonNeumann e de Moore;
  \item Criar interpretações para os dados obtidos;
  \item Buscar padrões recorrente e modelar relações com base neles;
  \item Buscar padrões de caos nos dados obtidos.
\end{enumerate}

% ----------------------------------------------------------
% PARTE
% ----------------------------------------------------------
%\part{O estudo}
% ----------------------------------------------------------

\chapter{Metodologia}

\section{Estudos iniciais}

O primeiro passo para este estudo foi conhecer a produção científica sobre autômatos celulares na área da Física. Para isso foram lidos alguns artigos, incluindo:
\begin{enumerate}
    \item \citeonline{lazzari}: “Estudo De Um Autômato Celular Para Modelar Ciclos De Expansão E Contração (“Boom E Burst”)”;
    \item \citeonline{klaus}: “Dinâmica de padrões em autômatos celulares com inércia”;
    \item \citeonline{stauffer}: “Adjustment and social choice”;
\end{enumerate}
Também foram verificadas implementações de autômatos celulares  com o software de simulações científicas Netlogo \cite{netlogo}. Ainda foi feita a leitura de alguns artigos sobre estatística \cite{statistics} e correlação para auxiliar na interpretação dos dados gerados pelas simulações. Adicionalmente, com o objetivo de obter conhecimento para criar implementações de autômatos celulares na linguagem de programação C, foram  lidos 19 capítulos do livro “C programming: A modern approach, 2nd Edition” do autor K.N. King \cite{king}.
    
\section{Modelagem e implementação de autômatos}

O primeiro desenvolvimento foi a percepção de características recorrentes nos autômatos celulares observados nos Estudos Iniciais.
A seguinte lista de características de autômatos celulares foi produzida com base nos artigos e software explorados:\\
\underline{Número de estados:} Quantos estados cada célula pode assumir.\\
\underline{Competitividade ou simbiose entre estados:} se a existência de um estado favorece ou inibe, nas proximidades, a existência de estados diferentes.\\
\underline{Vizinhança interna:} a região que é considerada interna a cada célula.\\
\underline{Vizinhança externa:} a região que é considerada externa a cada célula, apesar de ser próxima o suficiente para afetar diretamente o comportamento da célula.\\
\underline{Determinismo ou estocasticidade:} se a atualização das células ocorre de maneira previsível ou aleatoriamente.\\
\underline{Topologia do espaço:} como as bordas ou células estão conectadas.\\
\underline{Geometria do espaço:} formato das células e organização delas no espaço, caso aplicável.\\
\underline{Regras para atualização das células:} regras que determinam qual será o estado da célula no próximo passo ou ciclo da simulação.\\
\underline{Superposição de estados:} se é possível que cada célula tenha mais de um estado ao mesmo tempo.\\
\underline{Propriedades intrínsecas a cada célula:} características únicas a cada célula, além do estado.

O autômato celular não-homogêneo (InCA) de \citeonline{stauffer} foi modelado utilizando as características acima, resultando na descrição abaixo:\\
\underline{Número de estados:} Dois estados, -1 e 1.\\
\underline{Competitividade ou simbiose entre estados:} competitividade.\\
\underline{Topologia do espaço:} quadrada de tamanho $L\times L$, onde $L$ é a \textit{largura da matriz} em unidades de números de células, com bordas fechadas. Células conectadas verticalmente e horizontalmente por vizinhança mais próxima, mas não diagonalmente..\\
\underline{Geometria do espaço:} malha quadrada.\\
\underline{Vizinhança interna:} quadrada de raio 0 (somente a própria célula).\\
\underline{Vizinhança externa:} Cruz com eixos paralelos à malha da geometria do espaço. Raio 1. Formato de +. Vizinhança mais próxima. Células nas posições Norte, Sul, Leste e Oeste em relação à célula considerada.\\
\underline{Determinismo ou estocasticidade:} estocasticidade porque, em cada passo, uma célula escolhida aleatoriamente é atualizada com um parâmetro de valor aleatório.\\
\underline{Propriedades intrínsecas a cada célula:} cada célula $\mathbf{x}$ tem um valor intrínseco $\lambda_\mathbf{x}$, chamado \textit{limiar}, que determina qual a menor soma dos estados das células da vizinhança externa necessária para que a célula fique no estado +1 caso ela seja atualizada.\\
\underline{Regras para atualização das células:}  dada uma célula $\mathbf{x}$, caso ela seja atualizada, o limiar $\lambda_\mathbf{x}$ aumenta caso a célula fique +1 e diminui caso a célula fique -1. O valor $|\Delta\lambda_\mathbf{x}|$ é gerado aleatoriamente e está entre $0$ e $q$, onde $q$ é o \textit{ajuste máximo de limiar}.\\
\underline{Superposição de estados:} Não, somente um estado por célula.\\

Em seguida, esse modelo de autômatos celular foi implementado na linguagem C. Para cada simulação, a matriz foi inicializada com valores $-1$ ou $+1$ dispostos aleatoriamente de acordo com uma semente de geração de números aleatórios (\textit{seed}) baseada no horário de execução do programa. Não houve controle sobre a densidade de cada estado. Para todas as células o limiar foi iniciado em um número entre $-q$ e $+q$, onde $q$ é o ajuste máximo de limiar. A Figura \ref{fig:matrizL100Ciclo0} exibe a matriz estado inicial gerada em uma execução do InCA implementado.

\begin{figure}
    \centering
    \includegraphics[width=5cm]{matrizL100Ciclo0.png}
    \caption{Representação gráfica da matriz de estados iniciais de uma simulação com $L=100$. Células pretas representam o estado $+1$ e células brancas representam o estado $-1$. A matriz de estados iniciais é gerada de forma aleatória, com semente baseada no horário de execução da simulação.}
    \label{fig:matrizL100Ciclo0}
\end{figure}

A implementação do InCA foi planejada para imprimir várias informações sobre a situação da matriz: Ciclo, Estado Médio da matriz, Limiar Médio da matriz, e Número de Aglomerados. Um \textit{Ciclo} foi definido como sendo igual a $L\times L$ atualizações aleatórias de células na matriz, já que esse seria o número de atualizações de células caso o sistema fosse determinístico. Diferentemente de \citeonline{stauffer}, que executou as simulações somente para vizinhanças de VonNeumann (as 4 células ortogonais à célula central, nas posições norte, sul, leste e oeste), nós também fizemos as simulações para vizinhanças de Moore, que considera as 8 células vizinhas de uma célula central.

\section{O algoritmo de contagem de aglomerados}

O grande diferencial desta implementação do InCA, em relação ao estudo de Stauffer, foi a utilização de um algoritmo de contagem de aglomerados de células com o mesmo estado. Esse algoritmo, ilustrado na Figura \ref{fig:contamination}, encontra todas as células em um mesmo aglomerado através de ``contaminações'' sucessivas de células com o mesmo estado que são vizinhas ortogonais entre si. Nós desenvolvemos este algoritmo porque o algoritmo de Hoshen-Kopelman, utilizado na pesquisa de Kramer, se mostrou muito ineficiente e demorado.

\begin{figure}
    \centering
    \includegraphics[width=5cm]{contamination.png}
    \caption{Algoritmo de contagem de aglomerados por contaminação de células ortogonalmente vizinhas e com o mesmo estado. Cada aglomerado é contaminado a partir de uma primeira célula até que não existam mais células a serem contaminadas. Na figura, a primeira célula é a central (em vermelho). Em seguida, as quatro células adjacentes a essa célula são contaminadas. No passo seguinte as oito células adjacentes a essas quatro células são contaminadas. O processo é repetido até que não existam mais células a serem contaminadas.}
    \label{fig:contamination}
\end{figure}

 Definiremos, matematicamente, uma \index{contaminação}\textbf{contaminação} como: dado um conjunto $C_0$ de células com o mesmo estado numa matriz ortogonal, a contaminação de $C_0$ é denotada por $\overline{C_0}$ e definida como sendo o conjunto que tem por elementos todas as células da matriz que têm alguma célula de $C_0$ ortogonalmente vizinha e estão no mesmo estado das células de $C_0$. 

 Utilizando a definição acima, definiremos o algoritmo de contagem de aglomerados de células \texttt{+1} em uma matriz finita assim: 
\begin{enumerate}
	\item crie uma matriz booleana auxiliar de mesma dimensão e tamanho que a matriz original de células e inicie ela com o valor \texttt{false};
	%\item passando por cada célula da matriz original, verifique se o valor na posição correspondente na matriz auxiliar está em \texttt{true} ou o estado da célula na matriz original é \texttt{-1}. Se qualquer uma dessas interrogações for verdadeira, passe para a próxima célula da matriz original, repetindo este passo. Caso contrário, realize o próximo passo;
	%\item marque a posição correspondente da célula na matriz auxiliar como \texttt{true} e adicione a mesma a uma fila de espera (\textit{queue}). 
	\item passe por cada célula $c_i$ da matriz original. Se o estado da célula for \texttt{-1}, pule para a próxima célula. Se a posição correspondente dessa célula na matriz auxiliar tiver valor \texttt{true}, pule para a próxima célula. Caso contrário, execute o passo abaixo.
	\item dado o conjunto unitário inicial $C_0=\{c_i\}$, itere contaminações até que $C_n=\overline{C_n}$. Marque as posições equivalentes das células em $C_n$ na matriz auxiliar como \texttt{true}. Se o número de elementos em $C_n$ for maior que $1$, incremente o contador de número de aglomerados.
	\item repita o passo 2 até que todas as células da matriz tenham sido verificadas.
\end{enumerate}

O leitor com experiência em conjuntos perceberá que o conjunto de células contaminadas no passo $n$ é dado por $\overline{C_{n-1}}-C_{n-1}=C_n-C_{n-1}$ e que quando $C_{n-1}=\overline{C_{n-1}}=C_n$ teremos que o conjunto de células contaminadas é $C_n-C_{n-1}=C_n-C_n=\emptyset$. Portanto, a condição de término do processo de contaminação é equivalente ao momento em que não existem mais células a serem contaminadas.

Este algoritmo tem complexidade $O(N)$ (notação \textit{Big O}), onde $N$ é o número total de células na matriz. Isso pode ser facilmente verificado, considerando que cada célula $c$ será verificada no máximo $v(c)+1$, onde $v$ é uma função que retorna o número de células de uma dada célula. Para vizinhanças de VonNeumann, esse número é $5$ e para vizinhanças de Moore esse número é $9$, ambas em duas dimensões. Esse é um avanço considerável em comparação com o algoritmo de  Hoshen-Kopelman que, por análise prória do autor deste trabalho, demonstrou ter complexidade $O(N^2)$.

\section{Metodologias de análise}

As análises foram feitas com base nos dados de estado médio, limiar médio, e número de aglomerados em função do ciclo da simulação. Nesses gráficos foram buscados padrões recorrentes, como retas, parábolas ou elipses.

\subsection{A correlação entre estado médio e número de aglomerados}

O estudo da correlação entre estado médio e número de aglomerados foi motivado pela percepção de que os gráficos de número de aglomerados \textit{versus} estado médio sempre apresentavam inclinações com o mesmo sinal. Apesar de curto, este serviu de gatilho para o estudo de afinidades.

\subsection{Afinidade}

Como mencionado acima, foi percebido que o número de aglomerados estava relacionado ao estado médio da matriz. De fato, essa relação se mostrou aproximadamente linear para todos os conjuntos de dados obtidos, como ilustrado na Figura \ref{fig:exemploAfinidade}. 
\begin{figure}[h]
    \centering
    \includegraphics[width=.75\linewidth]{ICA-L50-q5_0-seed556358439-ClusterVsAvgState.png}
    \caption{Exemplo de como o número de aglomerados (eixo vertical) tende a ser linearmente dependente ao estado médio (eixo horizontal). Este gráfico contém 20000 pontos de dados obtidos numa simulação com ajuste máximo de limiar igual a $5.00$, largura de matriz igual a $50$ e vizinhança de Moore. Esse padrão foi observado em todas as outras medições realizadas. Fonte: produção própria.}
    \label{fig:exemploAfinidade}
\end{figure}
Considerando que todas as inclinações eram negativas, de modo que estados médios positivos implicavam em um número menor de aglomerados e estados médios negativos implicavam em um número maior de aglomerados, foi percebido que a inclinação da reta que melhor aproximava esses conjuntos de pontos representava a tendência de as células da matriz se aglomerarem, surgindo daí, o nome afinidade. Tecnicamente, definimos afinidade assim:

\textbf{Definição:} No InCA, seja $I_{q,L}$ a inclinação da reta que melhor aproxima os dados obtidos para número de aglomerados em função do estado médio em uma simulação com ajuste máximo de limiar igual a $q$ e largura de matriz $L$. Seja $Imin_L$ a menor inclinação encontrada entre todas as simulações para um dado $L$. Definimos a \textit{afinidade em função de $q$ e $L$} como sendo $A_{q,L}=-I_{q,L}/Imin_L$.

Esse padrão foi observado tanto para vizinhanças de VonNeumann como para vizinhanças de Moore, e os estudos foram feitos sobre como a afinidade se comporta quando os valores de $q$ e $L$ são alterados, além das diferenças entre os dados obtidos para ambas as vizinhanças.

Para a vizinhança de VonNeumann, as seguintes coleções de dados foram obtidas:\\
$\bullet$ Cinco conjunto de dados, com 20000 pontos cada, para cada combinação de $q$ e $L$ com $q\in\{$ 
  $0.1,$ $0.2,$ $0.3,$ $0.4,$ $0.5,$ $0.6,$ $0.7,$ $0.8,$ $0.9,$
  $1,$ $1.2,$ $1.4,$ $1.6,$ $1.8,$ 
  $2,$ $2.2,$ $2.4,$ $2.6,$ $2.8,$ 
  $3,$ $3.2,$ $3.4,$ $3.6,$ $3.8,$
  $4,$ $4.2,$ $4.4,$ $4.6,$ $4.8,$
  $5,$ $5.2,$ $5.4,$ $5.6,$ $5.8,$
  $6,$ $6.2,$ $6.4,$ $6.6,$ $6.8,$
  $7,$ $7.2,$ $7.4,$ $7.6,$ $7.8,$
  $8,$ $8.2,$ $8.4,$ $8.6,$ $8.8,$
  $9,$ $9.2,$ $9.4,$ $9.6,$ $9.8,$ $10,$
  $11,$ $12,$ $13,$ $14,$ $15,$ $16,$ $17,$ $18,$ $19,$ $20,$
  $22,$ $24,$ $26,$ $28,$ $30,$ $32,$ $34,$ $36,$ $38,$ $40,$
  $42,$ $44,$ $46,$ $48,$ $50,$ $52,$ $54,$ $56,$ $58,$ $60,$
  $65,$ $70,$ $75,$ $80,$ $85,$ $90,$ $95,$ $100,$ $110,$
  $120,$ $130,$ $140,$ $150,$ $160,$ $170,$ $180,$ $190,$
  $200,$ $220,$ $240,$ $260,$ $280,$ $300,$ $320,$ $340,$ $360,$ $380,$
  $400,$ $420,$ $440,$ $460,$ $480,$ $500,$ $520,$ $540,$ $560,$ $580,$
  $600,$ $620,$ $640,$ $660,$ $680,$ $700,$ $720,$ $740,$ $760,$ $780,$
  $800,$ $820,$ $840,$ $860,$ $880,$ $900,$ $920,$ $940,$ $960,$ $980,$
	$1000\}$ e $L\in \{50, 100, 250\}$.\\
	$\bullet$ Cinco conjunto de dados, com 10000 pontos cada, para $L=500$ e $q\in\{$
  $0.2,$ $0.4,$ $0.6,$ $0.8,$ $1,$
  $1.2,$ $1.4,$ $1.6,$ $1.8,$	$2,$
  $2.2,$ $2.4,$ $2.6,$ $2.8,$ $3,$
  $3.2,$ $3.4,$ $3.6,$ $3.8,$ $4,$
  $4.2,$ $4.4,$ $4.6,$ $4.8,$ $5,$
  $5.2,$ $5.4,$ $5.6,$ $5.8,$ $6,$
  $6.2,$ $6.4,$ $6.6,$ $6.8,$ $7,$
  $7.2,$ $7.4,$ $7.6,$ $7.8,$ $8,$
  $8.2,$ $8.4,$ $8.6,$ $8.8,$ $9,$
  $9.2,$ $9.4,$ $9.6,$ $9.8,$ $10$
	$\}$. \\
	Esse intervalo de $q$ entre 0 e 10 foi escolhido porque o principal fenômeno estudado acontecia entre esses valores. O tempo de simulação também foi um fator importante para a redução dos parâmetros utilizados (para $L=500$ a simulação chegou a durar mais de 12 horas, mesmo com parâmetros reduzidos).

	Já para a vizinhança de Moore, foram obtidas as seguintes coleções de dados:\\
	$\bullet$ Cinco conjuntos de 20000 pontos que considerou $L=50$ e $q\in\{$ 
  $0.1,$ $0.2,$ $0.3,$ $0.4,$ $0.5,$ $0.6,$ $0.7,$ $0.8,$ $0.9,$
  $1,$ $1.2,$ $1.4,$ $1.6,$ $1.8,$ 
  $2,$ $2.2,$ $2.4,$ $2.6,$ $2.8,$ 
  $3,$ $3.2,$ $3.4,$ $3.6,$ $3.8,$
  $4,$ $4.2,$ $4.4,$ $4.6,$ $4.8,$
  $5,$ $5.2,$ $5.4,$ $5.6,$ $5.8,$
  $6,$ $6.2,$ $6.4,$ $6.6,$ $6.8,$
  $7,$ $7.2,$ $7.4,$ $7.6,$ $7.8,$
  $8,$ $8.2,$ $8.4,$ $8.6,$ $8.8,$
  $9,$ $9.2,$ $9.4,$ $9.6,$ $9.8,$ $10,$
  $11,$ $12,$ $13,$ $14,$ $15,$ $16,$ $17,$ $18,$ $19,$ $20,$
  $22,$ $24,$ $26,$ $28,$ $30,$ $32,$ $34,$ $36,$ $38,$ $40,$
  $42,$ $44,$ $46,$ $48,$ $50,$ $52,$ $54,$ $56,$ $58,$ $60,$
  $65,$ $70,$ $75,$ $80,$ $85,$ $90,$ $95,$ $100,$ $110,$
  $120,$ $130,$ $140,$ $150,$ $160,$ $170,$ $180,$ $190,$
  $200,$ $220,$ $240,$ $260,$ $280,$ $300,$ $320,$ $340,$ $360,$ $380,$
  $400,$ $420,$ $440,$ $460,$ $480,$ $500,$ $520,$ $540,$ $560,$ $580,$
  $600,$ $620,$ $640,$ $660,$ $680,$ $700,$ $720,$ $740,$ $760,$ $780,$
  $800,$ $820,$ $840,$ $860,$ $880,$ $900,$ $920,$ $940,$ $960,$ $980,$
  $1000\}$. \\
	$\bullet$ Cinco conjuntos de 10000 pontos que considerou $L\in\{$ $100,$ $250,$ $500$ $\}$  e $q\in\{$ 
  $0.2,$ $0.4,$ $0.6,$ $0.8,$
  $1,$ $1.2,$ $1.4,$ $1.6,$ $1.8,$ 
  $2,$ $2.2,$ $2.4,$ $2.6,$ $2.8,$ 
  $3,$ $3.2,$ $3.4,$ $3.6,$ $3.8,$
  $4,$ $4.2,$ $4.4,$ $4.6,$ $4.8,$
  $5,$ $5.2,$ $5.4,$ $5.6,$ $5.8,$
  $6,$ $6.2,$ $6.4,$ $6.6,$ $6.8,$
  $7,$ $7.2,$ $7.4,$ $7.6,$ $7.8,$
  $8,$ $8.2,$ $8.4,$ $8.6,$ $8.8,$
  $9,$ $9.2,$ $9.4,$ $9.6,$ $9.8,$ $10,$
  $11,$ $12,$ $13,$ $14,$ $15,$ $16,$ $17,$ $18,$ $19,$ $20,$
  $22,$ $24,$ $26,$ $28,$ $30,$ $32,$ $34,$ $36,$ $38,$ $40,$
	$42,$ $44,$ $46,$ $48,$ $50$ $\}$\\
	$\bullet$ Obtenção de cinco conjuntos de 10000 pontos, para uma simulação que utilizou a vizinhança de Moore (8 células adjacentes), e considerou $L=500$  e $q\in\{$ 
  $0.2,$ $0.4,$ $0.6,$ $0.8,$
  $1,$ $1.2,$ $1.4,$ $1.6,$ $1.8,$ 
  $2,$ $2.2,$ $2.4,$ $2.6,$ $2.8,$ 
  $3,$ $3.2,$ $3.4,$ $3.6,$ $3.8,$
  $4,$ $4.2,$ $4.4,$ $4.6,$ $4.8,$
  $5,$ $5.2,$ $5.4,$ $5.6,$ $5.8,$
  $6,$ $6.2,$ $6.4,$ $6.6,$ $6.8,$
  $7,$ $7.2,$ $7.4,$ $7.6,$ $7.8,$
  $8,$ $8.2,$ $8.4,$ $8.6,$ $8.8,$
  $9,$ $9.2,$ $9.4,$ $9.6,$ $9.8,$ $10$ $\}$
\subsection{Caos}

Considerando que computadores geram números aleatórios a partir de uma \textit{seed}, foi verificada a presença de caos utilizando o seguinte método:
  \begin{enumerate}
    \item Geração da matriz inicial com uma \textit{seed} $S1$ (gerada com a função de números aleatórios de uma calculadora Casio);
    \item Inversão do sinal do estado e do limiar de $n$ células em posições aleatórias geradas com uma \textit{seed} baseada no horário mundial;
    \item Execução da simulação para 10000 ciclos utilizando $L=100$, limiar igual a $q$ e uma \textit{seed} igual a $S2$ (também gerada em uma Casio);
  \end{enumerate}
	Esse procedimento foi feito para $n\in\{0, 1, 10, 100, 1000, 10000\}$ e $q\in\{0.1, 1,$ $2,$ $3,$ $4,$ $5,$ $6,$ $7,$ $8,$ $9,$ $10\}$.\\
  Para a vizinhança de VonNeumann foram utilizadas as \textit{seeds} $S1=263598624$ e $S2=33386826$. Para a vizinhança de Moore foram utilizadas as \textit{seeds} $S1=156501936$ e $S2=2376222$. Representações gráficas dessas sementes estão dispostas nas Figuras \ref{fig:matrizInicialCaosMoore} e \ref{fig:matrizInicialCaosVonNeumann}.
  \begin{figure}[h]
    \subfloat[Matriz estado. Em branco o estado \texttt{-1}. Em preto o estado \texttt{+1}.]
    {
      \includegraphics[width=0.45\textwidth]{seed156501936.png}
    }\hfill
    \subfloat[Matriz limiar normalizada. Em branco o limiar mais alto. Em preto o limiar mais negativo.]
    {
      \includegraphics[width=0.45\textwidth]{seed156501936thres.png}
    }
    \caption{Matrizes iniciais geradas com a \textit{seed} 156501936. Essas matrizes foram utilizadas em todas as simulações de estudo de caos com vizinhança de Moore.}
    \label{fig:matrizInicialCaosMoore}
  \end{figure}

  \begin{figure}
    \subfloat[Matriz estado. Em branco o estado \texttt{-1}. Em preto o estado \texttt{+1}.]
    {
      \includegraphics[width=0.45\textwidth]{seed263598624.png}
    }\hfill
    \subfloat[Matriz limiar normalizada. Em branco o limiar mais alto. Em preto o limiar mais negativo.]
    {
      \includegraphics[width=0.45\textwidth]{seed263598624thres.png}
    }
    \caption{Matrizes iniciais geradas com a \textit{seed} 263598624. Essas matrizes foram utilizadas em todas as simulações de estudo de caos com vizinhança de VonNeumann.}
    \label{fig:matrizInicialCaosVonNeumann}
  \end{figure}
	

  A verificação da existência de caos foi feita considerando o coeficiente de Lyapunov
  \begin{align}
    \lambda = \dfrac{1}{n} \mathrm{ln}\left|\dfrac{f^n(x+\varepsilon) - f^n(x_0)}{\varepsilon}\right| = \dfrac{1}{n} \mathrm{ln}(\Delta)
  \end{align}
  Porém, considerando que $\mathrm{ln}(0)=-\infty$ e que computadores não lidam muito bem com infinito, foram feitas as seguintes observações com o intuito de desenvolver um coeficiente equivalente:
  \begin{enumerate}
    \item $\forall \Delta > 0$ tem-se que $\lambda < 0 \Leftrightarrow \mathrm{e}^\lambda < 1 \Leftrightarrow \Delta^\frac{1}{n} < 1 \Leftrightarrow \Delta^\frac{1}{10} < 1 \Rightarrow$ sistema estável 
    \item $\forall \Delta > 0$ tem-se que $\lambda = 0 \Leftrightarrow \mathrm{e}^\lambda = 1 \Leftrightarrow \Delta^\frac{1}{n} = 1 \Leftrightarrow \Delta^\frac{1}{10} = 1$ 
    \item $\forall \Delta > 0$ tem-se que $\lambda > 0 \Leftrightarrow \mathrm{e}^\lambda > 1 \Leftrightarrow \Delta^\frac{1}{n} > 1 \Leftrightarrow \Delta^\frac{1}{10} > 1 \Rightarrow$ sistema caótico 
  \end{enumerate}
 (O número 10 foi tomado arbitrariamente com a finalidade de diminuir números grandes). Portanto, para demonstrar que existe caos no sistema estudado é suficiente mostrar que $\Delta^\frac{1}{10} > 1$. 
 
 A verificação da quantidade de caos foi feita listando os valores médios de $\Delta^\frac{1}{10}$ para matrizes iniciais que tiveram entre $0.01\%$ a $100\%$ de suas células modificadas, para ambos os tipos de vizinhanças.



\chapter{Resultados e Análises}

\section{A interpretação por Volatilidade}

Diferentemente de \citeonline{stauffer}, que utilizou o \textit{Inhomogenous Cellular Automata} visando relações com o mercado financeiro, nós fizemos uma interpretação baseada na definição de \citeonline{anbima3}:
\begin{citacao}
  Em finanças, chamamos de \index{volatilidade}\textbf{volatilidade} o grau de variação dos preços de um ativo em
um determinado período de tempo, medido pelo conceito estatístico de desvio-padrão dos
retornos logarítmicos.Em finanças, chamamos de volatilidade o grau de variação dos preços de um ativo em
um determinado período de tempo, medido pelo conceito estatístico de desvio-padrão dos
retornos logarítmicos. \cite{anbima3}
\end{citacao}

No InCa, a velocidade máxima com que cada célula consegue mudar de estado é determinada, através do limiar interno a ela, pelo ajuste máximo de limiar $q$. Se redefinirmos $q$ como sendo sendo um indicador global de volatilidade, o estado de cada célula como \texttt{+1}=comprando e \texttt{-1}=vendendo, e o limiar de cada célula como o valor que cada célula dá ao ``produto'' do nosso sistema, então podemos entender as células como agentes em um mercado financeiro e, se essas hipóteses estiverem corretas, esse modelo será capaz de imitar fenômenos observados na economia.

Antes de exibirmos os dados, consideremos o que acontece no mercado em função da volatilidade, segundo \citeonline{anbima3}:
\begin{citacao}
 {}[\dots] Uma medida do risco
associado ao investimento em um ativo é justamente o grau de variação do seu preço de
mercado. Comparando dois investimentos com o mesmo retorno esperado, faz mais sentido
escolher aquele que oferece a menor variabilidade em seu preço, ou seja, aquele que sujeita o
seu detentor a uma menor dispersão entre os preços observados ao longo do tempo. \cite{anbima3}
\end{citacao}
Portanto, assumindo que o preço está diretamente relacionado ao valor do produto, devemos observar variações mais rápidas no limiar médio do sistema (valor médio) quando o ajuste máximo de limiar (volatilidade) for maior em relação a uma outra simulação com volatilidade menor. E de fato, é isso o que acontece nas simulações, como está demonstrado nas Figuras \ref{fig:volatilidadeMoore}a-c para a vizinhança de Moore e nas Figuras \ref{fig:volatilidadeVonNeumann}a-c para a vizinhança de VonNeumann.

\begin{figure}
  \centering
  \subfloat[Com volatilidade igual a 0.2, são observados aproximadamente 3.5 comprimentos de onda.]{\includegraphics[width=0.67\textwidth]{ICA-L500-q0_200-seed1255341753-AvgStateAvgThresVsCycle.png}}

  \subfloat[9.5 comprimentos de onda são observados para volatilidade=0.6.]{\includegraphics[width=0.67\textwidth]{ICA-L500-q0_600-seed214482678-AvgStateAvgThresVsCycle.png}}
  
  \subfloat[A volatilidade 1 apresenta mais de 12 comprimentos de onda.]{\includegraphics[width=0.67\textwidth]{ICA-L500-q1_0-seed561875194-AvgStateAvgThresVsCycle.png}}
  \caption{Gráficos dos valores médios (\textit{Average value}) do estado médio (\textit{Average State}) e do Limiar Médio (\textit{Average Threshold}) em função do ciclo de simulação. Interpretando o limiar médio como o valor médio de um produto e o estado médio como comprando (positivo) ou vendendo (negativo), percebe-se que quanto maior a volatilidade (q), mais comprimentos de onda são observados no mesmo intervalo de ciclos para o valor médio. É notável como o valor (limiar) continua alto mesmo após as células começarem a vender (estado médio caindo).}
  \label{fig:volatilidadeMoore}
\end{figure}

\begin{figure}
  \centering
  \subfloat[Volatilidade 0.2 e aproximadamente 8 comprimentos de onda.]{\includegraphics[width=0.67\textwidth]{VonICA-L500-q0_200-seed479735081-AvgStateAvgThresVsCycle.png}}

  \subfloat[Volatilidade 0.6 e aproximadamente 19 comprimentos de onda.]{\includegraphics[width=0.67\textwidth]{VonICA-L500-q0_600-seed328662057-AvgStateAvgThresVsCycle.png}}
  
  \subfloat[Volatilidade 1 e aproximadamente 28 comprimentos de onda.]{\includegraphics[width=0.67\textwidth]{VonICA-L500-q1_000-seed145500508-AvgStateAvgThresVsCycle.png}}
  \caption{Gráficos dos valores médios do estado médio (\textit{Average State}) e do Limiar Médio (\textit{Average Threshold}) em função do ciclo (\textit{Cycle}) de uma simulação com vizinhança de VonNeumann. Novamente percebe-se que, quanto maior a volatilidade (q), mais comprimentos de onda são observados no mesmo intervalo de ciclos para o valor (limiar) médio. Essas oscilações são em maior número do que as observadas para a vizinhança de Moore. Igualmente nota-se que o valor (limiar) continua alto mesmo após as células (agentes) começarem a vender. }
  \label{fig:volatilidadeVonNeumann}
\end{figure}

Dessa análise foi percebido que, quanto menos vizinhos cada célula tem, mais rápido o sistema oscila, independentemente do valor do ajuste máximo de limiar $q$. Interpretando novamente para o mercado financeiro entenderíamos que quanto menos contatos um agente econômico tem, mais rápido o sistema econômico oscila. Equivalentemente, se os agentes econômicos têm mais contatos, o sistema oscila mais lentamente. Interpretamos essa descoberta assim: quanto mais contatos um agente econômico tem, mais ele será influenciado, tornando suas decisões mais parecidas com as dos seus agentes próximos e assim tornando as mudanças de opinião mais lentas.


\section{A análise da Afinidade}
Como explicado na seção de metodologia, foi constatado que o número de aglomerados e o estado médio tendem a estar relacionados linearmente, e para estudar esse fenômeno foi definido o conceito de Afinidade, sendo realizadas várias simulações.

Uma análise da afinidade para várias larguras de matrizes e vários valores de ajuste máximo de limiar revelou que os gráficos desses coeficientes angulares em função de $q$ apresentam todos um formato sigmóide, semelhante a uma função logística. Também foi verificado que a Afinidade depende da largura da matriz. Essas observações estão ilustradas nas Figuras \ref{fig:afinidades}a-b.
\begin{figure}
    \centering
    \subfloat[Afinidades das simulações com vizinhança de VonNeumann. Percebe-se um ponto crítico próximo de $q=2$.]{\includegraphics[width=0.67\textwidth]{afinidadeVonNeumann.png}}

    \subfloat[Afinidades das simulações com vizinhança de Moore. Percebe-se um ponto crítico próximo de $q=5$.]{\includegraphics[width=0.67\textwidth]{afinidadeMoore.png}}
    \caption{Gráficos das afinidades para vizinhanças de VonNeumann e de Moore. Em ambos os casos percebe-se que o crescimento do afinidade no ponto crítico é menor para larguras menores de matrizes, como pode ser observado entre os pontos para $L=50$ (\texttt{+}) e $L=500$ ($\square$).}
    \label{fig:afinidades}
\end{figure}

Devido ao crescimento diferente entre as afinidades para larguras diferentes de matrizes, foi feita a derivada da afinidade em função do ajuste máximo de limiar $q$. Os resultados estão exibidos nas Figuras \ref{fig:DerivadasAfinidades}a-b. Desses dados foi estimado o valor máximo da derivada em função da largura $L$, que gerou as Tabelas \ref{tab:maximosAfinidadeVonNeumann} e \ref{tab:maximosAfinidadeMoore}.

\begin{figure}
  \centering
  \subfloat[Derivadas das afinidades em função de $q$ para as simulações com vizinhança de VonNeumann. Novamente percebe-se o ponto crítico próximo a $q=2$. A derivada cresce e cai rapidamente em volta do máximo.]{\includegraphics[width=0.67\textwidth]{afinidadeDeriVonNeumann.png}}

  \subfloat[Derivadas das afinidades em função de $q$ para as simulações com vizinhança de Moore. O ponto crítico próximo de $q=5$ permanece. Para esta vizinhança é observado um crescimento lento da derivada até o máximo e uma queda brusca após o máximo.]{\includegraphics[width=0.67\textwidth]{afinidadeDeriMoore.png}}
  \caption{Gráficos das derivadas das Afinidades em função do ajuste máximo de limiar $q$ para várias larguras de matriz diferentes. Nestes gráficos fica mais perceptível a diferença de inclinação no ponto crítico para os diferentes tamanhos de matriz.}
  \label{fig:DerivadasAfinidades}
\end{figure}

\begin{table}
  \centering
  \caption{Posições x e y para os máximos das derivadas da Afinidade em função de $q$ para várias larguras de matrizes em simulações com vizinhança de VonNeumann. Valores maiores de $y$ representam pontos críticos mais bruscos no gráfico da afinidade em função de $q$. Como observado nos gráficos, os dados mostram que o ponto crítico acontece perto de $2$, mais precisamente por volta de 1.75 para os dados analisados.}
  \label{tab:maximosAfinidadeVonNeumann}
  \begin{tabular}{ccc}\toprule
    L	& Máximo $x$ (q)	& Maximo $y$ (d Afinidade / dq)\\\midrule\midrule
    50	& 1.60 &	0.760\\
    100	& 1.72 &	0.938\\
    250	& 1.82 &	1.241\\
    500	& 1.87 &	1.324\\\bottomrule
  \end{tabular}
\end{table}

\begin{table}
  \centering
  \caption{Posições x e y para os máximos das derivadas da Afinidade em função de $q$ para várias larguras de matrizes em simulações com vizinhança de Moore. Valores maiores de $y$ representam pontos críticos mais bruscos no respectivo gráfico da afinidade em função de $q$. Semelhante aos gráficos, o ponto crítico acontece por volta de 5 para as larguras analisadas.}
  \label{tab:maximosAfinidadeMoore}
  \begin{tabular}{ccc}\toprule
    L	& Máximo $x$ (q)	& Maximo $y$ (d Afinidade / dq)\\\midrule\midrule
    50	& 4.50	& 0.293\\
    100	& 4.90	& 0.371\\
    250	& 5.10	& 0.469\\
    500	& 5.22	& 0.572\\\bottomrule
  \end{tabular}
\end{table}

A interpretação da afinidade foi feita considerando o conceito de liquidez de mercado, que é definida por \citeonline{anbima4} como:
\begin{citacao}
  A liquidez de um ativo pode ser definida como a facilidade com que este ativo pode ser
comprado ou vendido no mercado a um preço adequado. Em outras palavras, a liquidez indica
justamente a velocidade com que um ativo pode ser negociado sem que a própria negociação
influencie no preço do ativo. \cite{anbima4}
\end{citacao}

Se considerarmos o ajuste de limiar $q$ como sendo a liquidez, as células da matriz como agentes mercadológicos, o estado das células como \texttt{+1}=comprando e \texttt{-1}=vendendo, e o limiar de cada célula como a facilidade que cada célula tem de vender (limiar positivo) ou comprar (limiar negativo) um produto, então os dados obtidos nos fornecem um resultado interessante: quanto maior a liquidez de um produto num mercado (conjunto de agentes comprando ou vendendo um mesmo produto), mais os agentes mercadológicos tendem a comprar ou vender em conjunto, passando por um ponto crítico no qual o mercado todo entra em sincronia, tendo agentes que comprar e vendem em conjunto. Isso acontece porque quanto maior a liquidez (q), maior a afinidade aumenta. Só que a afinidade é uma medida de quanto as células estão unidas (em aglomerados) e sincronizadas. Pontanto é válida a interpretação.

Também foi conjecturada uma aplicação ao mundo real com base no que foi discutido acima: ao passo em que um mercado cresce e a liquidez do produto comercializado aumenta, a quantidade de agentes negociando o produto também aumentará, aumentando a necessidade de centros de negociação cada vez maiores a fim de comportar tantos agentes. A nossa pesquisa sobre afinidade revelou que, apesar de esse crescimento ser suave para a maioria dos valores de liquidez, existirá um momento em que os agentes se unirão em um único e grande centro de negociações. 

De acordo com o histórico institucional do CEASA-PR disponível no site \url{http://www.ceasa.pr.gov.br/modules/conteudo/conteudo.php?conteudo=1} (acesso em 31 de maio de 2019): ``Com o crescimento dos centros urbanos do país, o processo de distribuição de produtos hortigranjeiros tornou-se mais complexo e oneroso, o que aliado à precariedade dos mercados tradicionais, suscitou a necessidade de aperfeiçoamento das estruturas de comercialização desses produtos.'' Isso está alinhado com os estudos realizados, sendo que no caso do CEASA-PR, o fator decisivo para a aglomeração ou ponto crítico foram os altos custos da alternativa. 

\section{Caos}

A verificação de caos foi feita como estudo adicional e foi feita utilizando a função $\Delta^\frac{1}{10}$ que provamos na metodologia ser equivalente, para os fins desta análise, ao coeficiente de Lyapunov. Lembrando que $\Delta^\frac{1}{10}>1$ implica em caos e $\Delta^\frac{1}{10}<-1$ implica em regimes estáveis, os dados obtidos estão dispostos nas Tabelas \ref{tab:mediaDeltaMoore} e \ref{tab:mediaDeltaVonNeumann}. As Figuras \ref{fig:caos}a-c mostram o espectro de caos (utilizando o $\Delta^\frac{1}{10}<-1$) obtido para as simulações com $q=0.1$ e vizinhança de Moore. A grande surpresa que surgiu neste estudo foi a observação de como a modificação de uma única célula entre 9999 outras foi capaz de gerar comportamentos tão diferentes a longo prazo, como foi observado na Figura \ref{fig:caos}-a. Os estudos que foram feitos utilizando mais de 100 células, ou 1\% das células, como ruído, apesar de terem sido executados com a mesma seriedade dos outros casos, tiveram um objetivo mais lúdico do que prático pois já era esperado que eles se comportariam de forma diferente em relação aos dados originais utilizados como base.


\begin{table}[h]
  \centering
  \caption{Valores médios para $\Delta^\frac{1}{10}$ obtidos na simulação com Vizinhança de Moore. Apesar de todos serem maiores que 1, indicando a presença de caos, alguns são muito próximos de $1$, indicando sistemas menos caóticos no intervalo de ciclos observado.}
  \label{tab:mediaDeltaMoore}
  \begin{tabular}{c ccccc}
    \toprule
    q & \multicolumn{5}{c}{Valor médio de $\Delta^\frac{1}{10}$ para $\varepsilon$ máximo de}\\
     & 0.01\% & 0.1\% & 1\% & 10\% & 100\%\\\midrule\midrule
    0.1 & 1.8794	& 1.8456	& 1.7018	& 1.4393	& 1.8960\\
    1.0 & 1.8958	& 1.7182	& 1.5883	& 1.4302	& 1.2938\\
    2.0 & 2.0125	& 2.3890	& 1.6698	& 1.6073	& 1.5662\\
    3.0 & 1.2384	& 1.7053	& 1.9184	& 1.4607	& 1.4430\\
    4.0 & 1.3701	& 2.4282	& 1.9308	& 1.7997	& 1.3555\\
    5.0 & 1.8896	& 1.6471	& 1.5051	& 2.2331	& 1.2264\\
    6.0 & 1.5401	& 1.3400	& 1.4389	& 1.0733	& 1.1573\\
    7.0 & 1.5848	& 1.4721	& 1.3281	& 1.1638	& 1.1869\\
    8.0 & 1.3449	& 1.3471	& 1.2113	& 1.2330	& 1.0450\\
    9.0 & 1.4383	& 1.4295	& 1.2436	& 1.0116	& 1.0921\\
    10.0& 1.3170	& 1.7878	& 1.2356	& 1.2074	& 1.2092\\\bottomrule
  \end{tabular}
\end{table}

\begin{table}[h]
  \centering
  \caption{Valores médios para $\Delta^\frac{1}{10}$ obtidos na simulação com Vizinhança de VonNeumann. Novamente, a maioria está muito acima de 1, indicando a presença de caos. Também, alguns estão bem próximos de $1$, indicando sistemas com grandes regiões menos caóticas no intervalo de ciclos observado.}
  \label{tab:mediaDeltaVonNeumann}
  \begin{tabular}{c ccccc}
    \toprule
    q & \multicolumn{5}{c}{Valor médio de $\Delta^\frac{1}{10}$ para $\varepsilon$ máximo de}\\
     & 0.01\% & 0.1\% & 1\% & 10\% & 100\%\\\midrule\midrule
    0.1 & 2.2097 &	1.2283 &	1.6002 &	1.5985 &	2.0912\\
    1.0 & 1.3265 &	1.8247 &	1.5717 &	1.3925 &	1.4241\\
    2.0 & 1.3792 &	1.4424 &	1.4064 &	1.3782 &	1.5158\\
    3.0 & 1.5689 &	1.4663 &	1.3149 &	1.1522 &	1.0118\\
    4.0 & 1.4678 &	1.4277 &	1.3314 &	1.1317 &	1.0245\\
    5.0 & 1.3191 &	1.7714 &	1.2601 &	1.0664 &	1.7707\\
    6.0 & 1.4932 &	1.7547 &	1.7581 &	1.0064 &	1.0259\\
    7.0 & 1.4846 &	1.3871 &	1.2438 &	1.0284 &	0.9679\\
    8.0 & 1.3058 &	1.3805 &	1.2361 &	1.1376 &	1.1212\\
    9.0 & 1.4717 &	1.3764 &	1.1491 &	1.0828 &	1.0164\\
    10.0& 1.3545 &	1.2322 &	1.3704 &	1.1948 &	0.9669\\\bottomrule
  \end{tabular}
\end{table}

\begin{figure}
  \centering
  \subfloat[Espectro do caos para uma simulação que teve 1 célula modificada entre as 10000 da matriz original (0.01\%). Percebe-se que o sistema se comportou quase que identicamente com o original durante os primeiros 1000 ciclos, ocorrendo uma ``quebra caótica'' no ciclo 1000.]{\hspace{2cm}\includegraphics[width=0.6\textwidth]{caosq01noise1.png}\hspace{2cm}}

  \subfloat[Espectro do caos para uma simulação que teve 10 entre 10000 células modificadas (0.1\%). É possível perceber, pelos pontos com $\Delta^\frac{1}{10}$ próximo de 0 nos ciclos entre 0 e 1000, que o sistema se comportou parcialmente regular durante os primeiros ciclos, ficando totalmente caótico a partir do ciclo 1000.]{\hspace{2cm}\includegraphics[width=0.6\textwidth]{caosq01noise10.png}\hspace{2cm}}
  
  \subfloat[Espectro do caos para uma simulação que teve 1\% das células modificadas. Este caso demonstrou um comportamento diferente dos anteriores, apresentando caos desde o começo e entrando em harmonia com os dados de base em alguns momentos, como é visualizável entre os ciclos 5000 e 6000.]{\hspace{2cm}\includegraphics[width=0.6\textwidth]{caosq01noise100.png}\hspace{2cm}}
  \caption{Exemplos de comportamentos caóticos obtidos para simulações com vizinhança de Moore, $L=100$ e $q=0.1$.}
  \label{fig:caos}
\end{figure}

\section{Fenômenos não estudados}

Durante a pesquisa foram encontrados vários gráficos com comportamentos interessantes, como os das Figuras \ref{fig:clusterthres}. Gráficos como esses aconteceram recorrentemente nas simulações ao comparar o número de aglomerados com o Limiar Médio. Apesar do comportamento interessante, não foram feitos estudos sobre esta manifestação da simulação por desconhecimento de ferramentas próprias para tratar esses dados.

\begin{figure}
  \centering
  \subfloat[Padrão quase toroidal observado ao comparar o número de aglomerados em função do limiar médio em uma simulação com vizinhança de Moore. Na imagem estão representados os 10000 pontos da simulação realizada.]{\includegraphics[width=0.67\textwidth]{ICA-L100-q4_599-seed593122627-ClusterVsAvgThres.png}}

  \subfloat[Padrão toroidal com um núcleo formado por 10000 pontos de uma simulação com vizinhança de Moore.]{\includegraphics[width=0.67\textwidth]{ICA-L250-q5_0-seed134642580-ClusterVsAvgThres.png}}
  \caption{Gráficos do Número de Aglomerados (\textit{Number of Clusters}) em função do Limiar Médio (\textit{Average Threshold}). Esses padrões foram muito comuns nos resultados das simulações realizadas, tanto para vizinhanças de Moore como para de VonNeumann, acontecendo todos na relação entre Número de Aglomerados e Limiar Médio.}
  \label{fig:clusterthres}
\end{figure}

% ----------------------------------------------------------
% Finaliza a parte no bookmark do PDF
% para que se inicie o bookmark na raiz
% e adiciona espaço de parte no Sumário
% ----------------------------------------------------------
\phantompart

% ---
% Conclusão
% ---
\chapter{Conclusão}
% ---

Com os trabalhos desenvolvidos neste semestre foi possível desenvolver habilidades de análise e reprodução de autômatos celulares (em especial o \textit{Inhomogenous Cellular Automata} de Stauffer). A grande surpresa aconteceram neste estudo: a primeira foi os recorrentes padrões lineares no caso dos gráficos do número de aglomerados em função do estado médio; a segunda aconteceu na observação do poder que a modificação de $0.01\%$ da matriz inicial de uma simulação consegue causar a longo prazo, não deixando dúvidas de que o sistema estudado é caótico. A aplicação em dois conceitos da economia, a volatilidade e a liquidez, também serviram quase que perfeitamente como exemplos da aplicação do sistema desenvolvido, sendo possível encontrar exemplos como o CEASA, cuja necessidade de criação corrobora com o ponto crítico nos gráficos da Afinidade em função da Liquidez. Não menos importante se mostrou a fertilidade do \textit{Inhomogenous Cellular Automata} desenvolvido inicialmente por \citeonline{stauffer}, que pode ser modificado de inúmeras formas, gerando resultados diferentes e interessantes, como foi o caso de nosso desenvolvimento utilizando vizinhança de Moore. Exemplos disso foram os padrões que se assemelham a toróides ou planetas com anéis encontrados nas simulações ao se comparar o número de aglomerados com o limiar médio. Além dos resultados obtidos neste trabalho, tão importante foram os aprendizados desenvolvidos na realização dele, tanto em desenvolvimento de projetos, como em programação, e interpretação de dados. O leitor que se sentir interessado em explorar o programa de simulações de autômatos celulares desenvolvido pode obtê-lo acessando \url{https://github.com/Ly54ndr0/CellularAutomataExplorer} (ainda em desenvolvimento).

% ----------------------------------------------------------
% ELEMENTOS PÓS-TEXTUAIS
% ----------------------------------------------------------
\postextual
% ----------------------------------------------------------

% ----------------------------------------------------------
% Referências bibliográficas
% ----------------------------------------------------------
\bibliography{referencias}

% ----------------------------------------------------------
% Glossário
% ----------------------------------------------------------
%
% Consulte o manual da classe abntex2 para orientações sobre o glossário.
%
%\glossary

% ----------------------------------------------------------
% Apêndices
% ----------------------------------------------------------

% ---
% Inicia os apêndices
% ---
%\begin{apendicesenv}

% Imprime uma página indicando o início dos apêndices
%\partapendices

% ----------------------------------------------------------
%\chapter{Quisque libero justo}
% ----------------------------------------------------------

%\lipsum[50]

% ----------------------------------------------------------
%\chapter{Nullam elementum urna vel imperdiet sodales elit ipsum pharetra ligula
%ac pretium ante justo a nulla curabitur tristique arcu eu metus}
% ----------------------------------------------------------
%\lipsum[55-57]

%\end{apendicesenv}
% ---


% ----------------------------------------------------------
% Anexos
% ----------------------------------------------------------

% ---
% Inicia os anexos
% ---
%\begin{anexosenv}

% Imprime uma página indicando o início dos anexos
%\partanexos

% ---
%\chapter{Morbi ultrices rutrum lorem.}
% ---
%\lipsum[30]

% ---
%\chapter{Cras non urna sed feugiat cum sociis natoque penatibus et magnis dis
%parturient montes nascetur ridiculus mus}
% ---

%\lipsum[31]

% ---
%\chapter{Fusce facilisis lacinia dui}
% ---

%\lipsum[32]

%\end{anexosenv}

%---------------------------------------------------------------------
% INDICE REMISSIVO
%---------------------------------------------------------------------
%\phantompart
%\printindex
%---------------------------------------------------------------------

\end{document}
